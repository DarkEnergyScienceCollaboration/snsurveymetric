\documentclass[preprint]{aastex}
\usepackage{amsmath,amssymb}
\usepackage{mathrsfs}
\usepackage{tikz}
\usetikzlibrary{shapes,arrows}
\bibliographystyle{plain}
%\bibliographystyle{plain}
\newcommand{\NOTE}[1]{{\color{red}[{\it #1}]}} %for notes in draft
%\newcommand{\NOTE}[1]{}                       %uncomment for no notes

\newcommand{\mcm}[1]{{\color{blue}[{\it MCM: #1}]}} %for notes from Marisa in draft
%\newcommand{\mcm}[1]{}                       %uncomment for no notes from Marisa

\newcommand\SNR{\mathit{SNR}}
\newcommand{\Var}{\mathrm{Var}}


\begin{document}

\title{SN~Ia Survey Assessment Metric}
\author{Alex Kim}
\begin{abstract}
We present a metric that can be used to assess the relative quality of a supernova rolling survey within
an experiment such as DES or LSST.
The metric can be calculated with just a few lines of code using
only a few survey properties and observing conditions as inputs.
\end{abstract}

%\tableofcontents
\section{The Metric}
Given the redshift depth and light-curve quality requirements of a supernova survey, a quality requirement (or goal) can be given
for the  signal-to-noise ratio for band $\alpha$ ($\SNR_\alpha$) for some fiducial magnitude $m_\alpha$, and a target cadence $\delta t$.
The parameter $\tau$ is a supernova property  (not of the survey) that describes the SN time evolution.
For a realized survey, the quality of a given visit $i$ observed in band $\alpha$ is efficiently expressed as the signal-to-noise ratio $\SNR_{\alpha, i}$
for the fiducial magnitude $m_\alpha$ and the dates of the visits $t_i$ over the duration of the survey $T$.  For the purposes
of the DES and LSST surveys under consideration, $\SNR_{\alpha, i}$ is per night rather than per exposure.
The time separation between visits is $\Delta t_i=t_{i+1}-t_i$ and $\min{(\{\Delta_t\}}) \gtrsim 1$ day.  The solid-angle of the survey field covered
by the observation is $\Omega$.  The number of observations in a season for band $\alpha$ is $N_\alpha$.


The proposed metric for the survey field is
\begin{multline}
	\Omega \left(T-(1+z_{max})T_0\right)   \\
	\times
	\sum_{\alpha \in Bands}	
		\left(
		\frac{\tau(1-e^{-\frac{\delta t}{\tau}})}{\delta t}\right)^{-1}		\left\{
			\sum_{i \in visits}		\left[ 		\min{\left(
				\frac{\SNR_{\alpha,i}^2}{\SNR_{\alpha}^{2} } ,1 \right)}
				\frac{\tau}{T}	\left(1-e^{-\Delta t_{\alpha,i}/\tau}\right)\right]
				-a\frac{\Var\left(\min{\left(
				\SNR_{\alpha,i},{\SNR_{\alpha} }  \right)}
				\right)}{\SNR_\alpha^2}
				\right\}.
\end{multline}

The metric is proportional the the survey area and duration.  There is a term that gives the effective $\SNR^2$ averaged
over the duration of the survey.  The effective $\SNR^2$ is described by the area of a sawtooth function: each night is represented by a tooth
whose height is set by its $\SNR^2$ and has a exponential decline in time which is truncated by  next observation.  There is a penalty term
for variance in the $\SNR$ across all observations.

The behavior of the metric is described in more detail in the following:
\begin{itemize}
\item To first order, the variance in the cosmological parameters decreases linearly with the number of supernova, which is proportional to the surveyed solid angle: $\Omega$.
\item The number of SNe with with required $T_0$ temporal coverage with continuous monitoring $T$: $T-(1+z_{max})T_0$.
\item The quality of each band is considered independently and then combined: $\sum_{\alpha \in Bands}$.
\item Each band is normalized relative to a perfect survey getting the target $\SNR$ every $\delta t$ days: $\frac{\tau(1-e^{-\frac{\delta t}{\tau}})}{\delta t}$.
\item The combined $\SNR^2$ of multiple measurements of the same signal is is the quadratic sum of the individual $\SNR$'s: $\sum_{i \in \alpha} \SNR_{\alpha,i}^2$.
\item Typically the light-curve requirements are designed to push the systematic-error limits.  In other words, obtaining $\SNR_{\alpha,i} > \SNR_{\alpha}$ does not give the benefits gained in the absence of  systematic uncertainties: $\max{\left(\frac{\SNR_{\alpha,i}^2}{\SNR_{\alpha}^{2} } ,1 \right)}$.
\item A supernova observation starts as being fresh but then turns stale as time goes on until it is supplanted by a new observation.  The ``staling'' of the observation
is modeled as $\exp{(-t/\tau)}$.  One observation contributes with a weight $\int_0^{\Delta t_i} \exp{(-t/\tau)}dt$: $\tau(1-e^{-\Delta t_{\alpha,i}/\tau})$.
\item There is an integral over $t$ that is normalized out: $1/T$.
\item The timescale of supernova evolution is shorter than the length of a season.  We seek uniformity of data quality within a supernova light curve and between light curves
of different supernovae: $-a\SNR_{\alpha}^{-2} \Var\left(\SNR_{\alpha,i}\right)$
\item $a$ is a parameter that describes the amount of negative impact of data non-uniformity.  Its value can be determined through simulation.  For now it is set to a value of 1
\end{itemize}

\section{Monte Carlo}
Monte Carlo simulations of surveys show the performance of the metric.  For this example we consider only a single band, $\tau=4$, $\SNR=20$, $N=20$, $a=1$.
The situation where the observations are uniformly spaced with constant signal to noise, shown as the narrow black line in
Figure~\ref{mc:fig}.  This survey outperforms all of the simulated surveys.
The Random Phase Monte Carlo keeps the same number of observations but randomizes their dates and is seen as the blue histogram.
The Random $\SNR$ Monte Carlo has uniformly spaced observations but randomizes $\SNR \sim \mathcal{N}(20,1)$. 
\begin{figure}[htbp] %  figure placement: here, top, bottom, or page
   \centering
   \plotone{fom.eps} 
   \caption{Distributions of the metric for different Monte Carlo simulations.  Uniformly spaced observations with constant signal to noise is shown as the narrow black line.
   The Random Phase Monte Carlo 
   keeps the same number of observations but randomizes their dates.  The Random $\SNR$ Monte Carlo has uniformly spaced observations but randomizes $\SNR \sim \mathcal{N}(20,1)$.  }
   \label{mc:fig}
\end{figure}

\section{Surveys}
\subsection{DES}
Obstac output files provide 

\section{Defects}
Two observations separated by a day and with very different $\SNR$'s, and then followed by a next observation with a time gap $>\tau$, should provide a very similar metric independent
of the order of the high- and low-$\SNR$ observation.  This expectation does not hold for this metric.
\end{document}
